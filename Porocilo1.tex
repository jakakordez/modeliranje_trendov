% To je predloga za poročila o domačih nalogah pri predmetih, katerih
% nosilec je Blaž Zupan. Seveda lahko tudi dodaš kakšen nov, zanimiv
% in uporaben element, ki ga v tej predlogi (še) ni. Več o LaTeX-u izveš na
% spletu, na primer na http://tobi.oetiker.ch/lshort/lshort.pdf.
%
% To predlogo lahko spremeniš v PDF dokument s pomočjo programa
% pdflatex, ki je del standardne instalacije LaTeX programov.

\documentclass[a4paper,11pt]{article}
\usepackage{a4wide}
\usepackage{fullpage}
\usepackage[utf8x]{inputenc}
\usepackage[slovene]{babel}
\selectlanguage{slovene}
\usepackage[toc,page]{appendix}
\usepackage[pdftex]{graphicx} % za slike
\usepackage{setspace}
\usepackage{color}
\definecolor{light-gray}{gray}{0.95}
\usepackage{listings} % za vključevanje kode
\usepackage{hyperref}
\renewcommand{\baselinestretch}{1.2} % za boljšo berljivost večji razmak
\renewcommand{\appendixpagename}{Priloge}

\lstset{ % nastavitve za izpis kode, sem lahko tudi kaj dodaš/spremeniš
language=Python,
basicstyle=\footnotesize,
basicstyle=\ttfamily\footnotesize\setstretch{1},
backgroundcolor=\color{light-gray},
}

\title{Modeliranje časovnih trendov z markovskimi verigami \\ \large  Poročilo izvorne kode (serijski algoritem)}
\author{Jaka Kordež \\ Anže Gregorc}
\date{\today}

\begin{document}

\maketitle

\section{Uvod}

Pri predmetu bomo spoznavali sisteme za vzporedno in porazdeljeno procesiranje. Izbrali smo problem, ki ga bomo v skupinah po dva tekom semestra nadgrajevali s pomočjo različnih pristopov za paralelno programiranje. Tokrat pa je poročilo namenjeno serijskem algoritmu, ki še ni paraleliziran. 

\section{Podatki}

Ker je naš namen v bližnji prihodnosti predvideti ceno določenega trga, sva vzela podatke valutnega para EUR/USD iz spletne strani \url{http://www.histdata.com/download-free-forex-historical-data/?/metatrader/1-minute-bar-quotes/EURUSD}.

\section{Metode}

Program deluje tako, da pri branju podatkov shrani razliko v ceni med zaporednima cenama trga. Nato te cene pretvori v stanja, ki se jih uporablja v markovskih verigah. Meja med stanji je določena eksponentno. Na primer: če imamo 6 stanj bodo meje med stanji za določeno razliko postavljene na -0.01; -0.0025; 0.0; 0.0025; 0.01. Stanja se potem vnašajo v markovsko verigo. S pomočjo drsečega okna se zadnjih 5 zaporednih podatkov pretvori v indeks stolpca matrike in prištejemo 1 v vrstico trenutnega stanja. Nato se matriko normaliriza  po vrsticah, da dobimo pravo markovsko verigo.


\section{Rezultati}



\end{document}
